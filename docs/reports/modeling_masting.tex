% Options for packages loaded elsewhere
\PassOptionsToPackage{unicode}{hyperref}
\PassOptionsToPackage{hyphens}{url}
\PassOptionsToPackage{dvipsnames,svgnames,x11names}{xcolor}
%
\documentclass[
  letterpaper,
  DIV=11,
  numbers=noendperiod]{scrartcl}

\usepackage{amsmath,amssymb}
\usepackage{iftex}
\ifPDFTeX
  \usepackage[T1]{fontenc}
  \usepackage[utf8]{inputenc}
  \usepackage{textcomp} % provide euro and other symbols
\else % if luatex or xetex
  \usepackage{unicode-math}
  \defaultfontfeatures{Scale=MatchLowercase}
  \defaultfontfeatures[\rmfamily]{Ligatures=TeX,Scale=1}
\fi
\usepackage{lmodern}
\ifPDFTeX\else  
    % xetex/luatex font selection
    \setmonofont[Scale=0.9]{Latin Modern Mono}
\fi
% Use upquote if available, for straight quotes in verbatim environments
\IfFileExists{upquote.sty}{\usepackage{upquote}}{}
\IfFileExists{microtype.sty}{% use microtype if available
  \usepackage[]{microtype}
  \UseMicrotypeSet[protrusion]{basicmath} % disable protrusion for tt fonts
}{}
\makeatletter
\@ifundefined{KOMAClassName}{% if non-KOMA class
  \IfFileExists{parskip.sty}{%
    \usepackage{parskip}
  }{% else
    \setlength{\parindent}{0pt}
    \setlength{\parskip}{6pt plus 2pt minus 1pt}}
}{% if KOMA class
  \KOMAoptions{parskip=half}}
\makeatother
\usepackage{xcolor}
\usepackage[top=20mm,left=20mm,bottom=20mm,right=20mm]{geometry}
\setlength{\emergencystretch}{3em} % prevent overfull lines
\setcounter{secnumdepth}{5}
% Make \paragraph and \subparagraph free-standing
\makeatletter
\ifx\paragraph\undefined\else
  \let\oldparagraph\paragraph
  \renewcommand{\paragraph}{
    \@ifstar
      \xxxParagraphStar
      \xxxParagraphNoStar
  }
  \newcommand{\xxxParagraphStar}[1]{\oldparagraph*{#1}\mbox{}}
  \newcommand{\xxxParagraphNoStar}[1]{\oldparagraph{#1}\mbox{}}
\fi
\ifx\subparagraph\undefined\else
  \let\oldsubparagraph\subparagraph
  \renewcommand{\subparagraph}{
    \@ifstar
      \xxxSubParagraphStar
      \xxxSubParagraphNoStar
  }
  \newcommand{\xxxSubParagraphStar}[1]{\oldsubparagraph*{#1}\mbox{}}
  \newcommand{\xxxSubParagraphNoStar}[1]{\oldsubparagraph{#1}\mbox{}}
\fi
\makeatother


\providecommand{\tightlist}{%
  \setlength{\itemsep}{0pt}\setlength{\parskip}{0pt}}\usepackage{longtable,booktabs,array}
\usepackage{calc} % for calculating minipage widths
% Correct order of tables after \paragraph or \subparagraph
\usepackage{etoolbox}
\makeatletter
\patchcmd\longtable{\par}{\if@noskipsec\mbox{}\fi\par}{}{}
\makeatother
% Allow footnotes in longtable head/foot
\IfFileExists{footnotehyper.sty}{\usepackage{footnotehyper}}{\usepackage{footnote}}
\makesavenoteenv{longtable}
\usepackage{graphicx}
\makeatletter
\def\maxwidth{\ifdim\Gin@nat@width>\linewidth\linewidth\else\Gin@nat@width\fi}
\def\maxheight{\ifdim\Gin@nat@height>\textheight\textheight\else\Gin@nat@height\fi}
\makeatother
% Scale images if necessary, so that they will not overflow the page
% margins by default, and it is still possible to overwrite the defaults
% using explicit options in \includegraphics[width, height, ...]{}
\setkeys{Gin}{width=\maxwidth,height=\maxheight,keepaspectratio}
% Set default figure placement to htbp
\makeatletter
\def\fps@figure{htbp}
\makeatother

\usepackage{bm}
\usepackage{xcolor}
\definecolor{mid}{HTML}{B97C7C}
\definecolor{light_teal}{HTML}{6B8E8E}
\usepackage{amsmath}
\KOMAoption{captions}{tableheading}
\makeatletter
\@ifpackageloaded{tcolorbox}{}{\usepackage[skins,breakable]{tcolorbox}}
\@ifpackageloaded{fontawesome5}{}{\usepackage{fontawesome5}}
\definecolor{quarto-callout-color}{HTML}{909090}
\definecolor{quarto-callout-note-color}{HTML}{0758E5}
\definecolor{quarto-callout-important-color}{HTML}{CC1914}
\definecolor{quarto-callout-warning-color}{HTML}{EB9113}
\definecolor{quarto-callout-tip-color}{HTML}{00A047}
\definecolor{quarto-callout-caution-color}{HTML}{FC5300}
\definecolor{quarto-callout-color-frame}{HTML}{acacac}
\definecolor{quarto-callout-note-color-frame}{HTML}{4582ec}
\definecolor{quarto-callout-important-color-frame}{HTML}{d9534f}
\definecolor{quarto-callout-warning-color-frame}{HTML}{f0ad4e}
\definecolor{quarto-callout-tip-color-frame}{HTML}{02b875}
\definecolor{quarto-callout-caution-color-frame}{HTML}{fd7e14}
\makeatother
\makeatletter
\@ifpackageloaded{caption}{}{\usepackage{caption}}
\AtBeginDocument{%
\ifdefined\contentsname
  \renewcommand*\contentsname{Table of contents}
\else
  \newcommand\contentsname{Table of contents}
\fi
\ifdefined\listfigurename
  \renewcommand*\listfigurename{List of Figures}
\else
  \newcommand\listfigurename{List of Figures}
\fi
\ifdefined\listtablename
  \renewcommand*\listtablename{List of Tables}
\else
  \newcommand\listtablename{List of Tables}
\fi
\ifdefined\figurename
  \renewcommand*\figurename{Figure}
\else
  \newcommand\figurename{Figure}
\fi
\ifdefined\tablename
  \renewcommand*\tablename{Table}
\else
  \newcommand\tablename{Table}
\fi
}
\@ifpackageloaded{float}{}{\usepackage{float}}
\floatstyle{ruled}
\@ifundefined{c@chapter}{\newfloat{codelisting}{h}{lop}}{\newfloat{codelisting}{h}{lop}[chapter]}
\floatname{codelisting}{Listing}
\newcommand*\listoflistings{\listof{codelisting}{List of Listings}}
\makeatother
\makeatletter
\makeatother
\makeatletter
\@ifpackageloaded{caption}{}{\usepackage{caption}}
\@ifpackageloaded{subcaption}{}{\usepackage{subcaption}}
\makeatother

\ifLuaTeX
  \usepackage{selnolig}  % disable illegal ligatures
\fi
\usepackage{bookmark}

\IfFileExists{xurl.sty}{\usepackage{xurl}}{} % add URL line breaks if available
\urlstyle{same} % disable monospaced font for URLs
\hypersetup{
  pdftitle={Modeling masting},
  colorlinks=true,
  linkcolor={blue},
  filecolor={Maroon},
  citecolor={Blue},
  urlcolor={Blue},
  pdfcreator={LaTeX via pandoc}}


\title{Modeling masting}
\usepackage{etoolbox}
\makeatletter
\providecommand{\subtitle}[1]{% add subtitle to \maketitle
  \apptocmd{\@title}{\par {\large #1 \par}}{}{}
}
\makeatother
\subtitle{From individual tree behavior to population-level synchrony}
\author{}
\date{}

\begin{document}
\maketitle


\section{The motivations behind the
project}\label{the-motivations-behind-the-project}

\begin{tcolorbox}[enhanced jigsaw, title=\textcolor{quarto-callout-tip-color}{\faLightbulb}\hspace{0.5em}{The ecological motivation}, coltitle=black, colbacktitle=quarto-callout-tip-color!10!white, breakable, opacityback=0, colback=white, bottomrule=.15mm, rightrule=.15mm, left=2mm, bottomtitle=1mm, colframe=quarto-callout-tip-color-frame, toptitle=1mm, toprule=.15mm, titlerule=0mm, arc=.35mm, leftrule=.75mm, opacitybacktitle=0.6]

Understanding the reproductive behavior that arises at the population
level requires to study individual trees' responses. We believe that
integrating reproductive biology (i.e.~biological \textbf{constraints})
and environmental \textbf{cues} at the \emph{individual} level can help
us investigating the drivers of the variability and synchronicity of
reproduction at the \emph{population} scale. Our final goal is to
understand to what extent climate change could really disrupt forest
regeneration---and provide better forecasts!

\end{tcolorbox}

\hfill\break

\begin{tcolorbox}[enhanced jigsaw, title=\textcolor{quarto-callout-tip-color}{\faLightbulb}\hspace{0.5em}{The technical motivation}, coltitle=black, colbacktitle=quarto-callout-tip-color!10!white, breakable, opacityback=0, colback=white, bottomrule=.15mm, rightrule=.15mm, left=2mm, bottomtitle=1mm, colframe=quarto-callout-tip-color-frame, toptitle=1mm, toprule=.15mm, titlerule=0mm, arc=.35mm, leftrule=.75mm, opacitybacktitle=0.6]

Bayesian magic! Our approach models the (latent) reproductive states of
each \emph{individual} tree, and the consequent amount of seed
production. The states explicitly encode the \textbf{constraints} that
shape tree reproduction -- in particular, the fact that most trees need
at least two years between flower bud differentiation and fruit
maturation. We do not standardize away the complexities of
individual-level reproduction (e.g.~with normalized stand-level indices
between 0 and 1\ldots). On the contrary, modeling tree-level
reproduction allows us to obtain \emph{population} estimates -- and thus
direct inferences on the population-level variability and synchronicity!
And the cherry on top: we incorporate climatic \textbf{cues} at
different key moments of the reproductive cycle.

\end{tcolorbox}

\newpage

\section{The model}\label{the-model}

\subsection{Previous models and
limitations}\label{previous-models-and-limitations}

The observations we have---whether for an individual tree or a seed
trap---are a time series of yearly seed counts \(y_i\).

\begin{center}
\includegraphics{modeling_masting_files/figure-pdf/unnamed-chunk-3-1.pdf}
\end{center}

The following quote illustrates the current approach of statistical
model of masting:

\begin{quote}
``We fitted a zero-inflated, negative binomial mixed model to the annual
number of initiated seeds in each tree, with fixed factors that included
summer temperatures in 1 and 2 years before seedfall, {[}\ldots{]} and
seed production in the previous year to account for possible resource
depletion. {[}\ldots{]} We included {[}\ldots{]} a first-order temporal
autocorrelation structure.''
\end{quote}

If we ignore the zero-inflated part, we could write this model as
something like:

\[y_i \sim \text{NegBin}(\mu_i, \phi)\]
\[\mu_i = \alpha + \beta^\text{summer}_{n-1}.X^\text{summer}_{i-1} + \beta^\text{summer}_{n-2}.X^\text{summer}_{i-2} + \beta^\text{seed}_{n-1}.y_{i-1}+ \rho.\epsilon_{i-1}+\epsilon_i\]
\[\epsilon_i \sim \mathcal{N}(0,1)\] The idea is to introduce some
dependency between the seed counts:

\begin{center}
\includegraphics{modeling_masting_files/figure-pdf/unnamed-chunk-4-1.pdf}
\end{center}

The inclusion of both a lagged response of
\(\beta^\text{seed}_{n-1}.y_{i-1}\) and a first-order autocorrelation
structure on the residuals \(\rho.\epsilon_{i-1}\) seems\ldots{} quite
tricky.

But more importantly, the effect of summer temperature does not directly
depends on the previous state. In other words, whatever the value of
\(\beta^\text{seed}_{n-1}.y_{i-1}\), a tree that experiences a warm
summers would be predicted to have an increasing seed
production---regardless of the previous reproductive state.

\begin{center}
\includegraphics{modeling_masting_files/figure-pdf/unnamed-chunk-5-1.pdf}
\end{center}

This issue arises because we does not explicitly simulate the
reproductive state of an individual tree. The model does not allow to
differentiate effects of climate during a non-masting year or a masting
year.\\
Moreover, without modeling individual-level states, it is not possible
to infer reproductive behavior at the population scale---including the
degree of potential synchrony.

\subsection{A new approach grounded in
biology}\label{a-new-approach-grounded-in-biology}

Researchers have worked on the reproductive biology of trees for
decades---and in particular fruit trees. In parallel, people working on
masting have defined characteristics of masting and developed clear
hypotheses about the evolutionary mechanisms that could cause masting. ~
There is thus a great opportunity to develop more generative models,
that explicitly incorporate biological knowledge of flower and fruit
development in order to better understand how masting works.

To understand the reproductive behavior that arises at the population
level, we need to model individual trees' reproduction. At the tree
level, floral buds are initiated the year before flowering---in the same
time as fruits of the current year start developing. During a large crop
year (an ``on''-year), the presence of many fruits depress flower
initiation because of hormonal inhibition and resource trade-off. Thus,
the next year will likely be an ``off''-year. This behavior is called
\textbf{alternate bearing}. It is not always a 2-year cycle, as an
``on''-year can be followed by several ``off''-years. This clearly
defines two states for a tree.

\begin{center}
\includegraphics{modeling_masting_files/figure-pdf/unnamed-chunk-6-1.pdf}
\end{center}

Each state \(z_{i+1}\) is dependent \textbf{only on the preceding state}
\(z_{i}\).\\
So, with some parameters \(\theta\), the probability of observing some
sequences of the two first states is:
\[p(z_1, z_2 ~ | ~\theta) = p(z_1) \times p(z_2 ~ | ~ z_1,\theta)\] And
then, adding the next state:
\[p(z_1, z_2, z_3 ~ | ~\theta) = p(z_1) \times p(z_2 ~ | ~ z_1,\theta) \times p(z_3 ~ | ~ z_2, \theta)\]
And so on\ldots{}

Each observed seed count at each step arises from the state of the tree.

\begin{center}
\includegraphics{modeling_masting_files/figure-pdf/unnamed-chunk-7-1.pdf}
\end{center}

This would give us this likelihood for the two first seed counts:

\[p(y_1, y_2, z_1, z_2 ~ | ~\theta) = p(z_1) \times p(y_1 ~ | ~ z_1,\theta) \times p(z_2 ~ | ~ z_1,\theta) \times p(y_2 ~ | ~ z_2,\theta)\]




\end{document}
