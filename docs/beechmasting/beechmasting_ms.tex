\documentclass[11pt]{article}
\usepackage[top=1.00in, bottom=1.0in, left=1in, right=1in]{geometry}
\renewcommand{\baselinestretch}{1.1}
\usepackage{graphicx}
\usepackage{natbib}
\usepackage{amsmath}
\usepackage{hyperref}

\begin{document}
\renewcommand{\refname}{\CHead{}}

\title{Climatic drivers and intrisic biological processes shape masting dynamics...} 
\date{\today}
\author{Victor Van der Meersch, Mike Betancourt, \& EM Wolkovich}
\maketitle


\begin{enumerate}

\item Intro (all forests $\rightarrow$ stand sync $\rightarrow$ ind. trees)
%\begin{itemize}

%\item Widespread meltdown fears\\
%$\rightarrow$ maybe mention invasion, homogeneization, pest outbreaks...\\
%$\rightarrow$ regeneration shifts... masting!

The acceleration of climate change is predicted to have abrupt ecological effects worldwide. Rapid shifts to novel climate conditions with more extreme events have the potential to disrupt key ecological processes. Feedback mechanisms could break down resilience and potentially drive ecosystems toward critical transitions. In particular, many forest ecosystems are show sign of increased sensitivity to biotic and abiotic disturbances. Forests could adapt if they can rely on their regeneration capacity, which promotes post-disturbance recolonization and the establishment of individuals better adapted to new conditions. 

% \item Meltdown and masting
% \begin{itemize}
% \item simple definition of masting
% \item ...
% \item end paragraph on masting being hypothesized as critical to regeneratino via seed predator overwhelmed... or pollen etc. and this could have big effects!
% \end{itemize}
Regeneration in many temperate and tropical forests depends on tree species that have a high variability in reproduction across years, with most individuals of a population reproducing synchrounously. These two characterisics---variability and synchronicty---define masting, which is hypothesized to have strong fitness benefits, mostly because it could allow a higher proportion of seeds to escape predation. Masting could also favor greater pollen exchange and genetic outcrossing across individuals, favoring the production of seedlings that may be better adapted to new climatic conditions. However, disruption of this reproductive timing by climate change could trigger cascading effects on forest resilience.

% \item How does masting work? Synchrony
% \begin{itemize}
% \item individual trees need to be synchronized within certain distance (but maybe not further...)
% \item population-level characteristics! So how do individual treews do it together?
% \end{itemize}
... Population-level masting arises from the reproductive behavior of individual trees.

% \item How do individual trees cue to mast?
% \begin{itemize}
% \item what is a good year environmentally... warm summer, no frost
% \item but that doesn't cause synchrony! No cues can cause 2-year cycles or other cycles...
% \item however, most trees take 2 years to make buds, and have alternate bearing
% \item (Conceptual figure)
% \end{itemize}


% \item Combo of cues + constraints\\
% $\rightarrow$ could explain what we see?
% \begin{itemize}
% \item year before need warm summer...
% \end{itemize}

% \item So ACC could really screw this up
% $\rightarrow$ but to know this, we need to model the individuals!

% \end{itemize}

\item Results and discussion

\begin{itemize}

\item We built a model that matches conceptual figure
\begin{itemize}
\item alternate states (latent)
\item states encode constraints 
\item tree level estimates lead to stand estimates!
\item and we added climate
\end{itemize}

\item Model identifies 2 states (here, figure with the two distributions)
\begin{itemize}
\item masting is real! Mirror the intro
\item some level of synchrony within stands
\item say how often they transition in average conditions...
\end{itemize}

\item Climate impacts on masting (figure of climate effects)
\begin{itemize}
\item warm summer increase transition
\item frost decrease number of seeds 
\item no efect of spring (supp mat)
\end{itemize}

\item Our projections vs current studies
\begin{itemize}
\item current studies: ACC leads to more seeds via more masting
\item but even if you drive warming way upp you still get a plateau
\item this even happens with summer temp effect on M to M (figure proj)
\item To actually ahve a breakdown, we would nee the parameter valeu on M to M to be at least as important as NM to M
\end{itemize}

\item How constraints prevent breakdown!
\begin{itemize}
\item ...
\end{itemize}

\item But synchrony does appear to go down
\begin{itemize}
\item Review previous results and overall figure
\item these years look less synchrone\dots
\item but here, it could be driven both by within and between asynchrony
\item (what level of between-stand synchrony predict..?)
\item evolutionary benefits of mating depends on scale of synchrony \\
$rightarrow$ which scale depends on which evolutionnary model you consider, but for seed predators... should be quite small (foraging distance = X km)
\end{itemize}

\item Asynchrony indeed driven by multiple factors
\begin{itemize}
\item within between
\item discuss results... maybe figure with \%?
\end{itemize}

\item What drives synchrony? 
\begin{itemize}
\item bad years could act as precise cue, and with biol. constraints it would explain the following synchrony 
\item how ACC could change those dynamics, and on which scale?
\item (Unclear how breakdown at tree and then at stand level?)
\item basically, we need to figure out the biology useful for predictions with ACC
\end{itemize}

\end{itemize}


\end{enumerate}

\end{document}

